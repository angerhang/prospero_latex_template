\documentclass[11pt,letterpaper]{article}

\usepackage{typearea}
\paperwidth 8.5in \paperheight 11in
\typearea{15}
\usepackage{algorithm2e}
\usepackage{natbib}
\usepackage{latexsym,graphicx}
\usepackage{amsmath,amssymb,enumerate}
\usepackage{amsthm}
\usepackage{graphicx}
\usepackage{float}
\usepackage{xspace}
%\usepackage[showonlyrefs,showmanualtags]{mathtools}
\usepackage{bm}
\usepackage{pifont}% http://ctan.org/pkg/pifont
\usepackage{ifpdf}
\usepackage{shadow,shadethm,color}
\usepackage[compact]{titlesec}
%\numberwithin{algorithm}{section}
\usepackage{listings}
\usepackage{color}

\usepackage{lipsum}                     % Dummytext
\usepackage{xargs}                      % Use more than one optional parameter in a new commands
\usepackage[pdftex,dvipsnames]{xcolor}  % Coloured text etc.
% 
\usepackage{ colortbl}
\definecolor{carnationpink}{rgb}{1.0, 0.65, 0.79}
\definecolor{Gray}{gray}{0.9}
\usepackage[first=0,last=9]{lcg}
\newcommand{\ra}{\rand0.\arabic{rand}}


\usepackage[colorinlistoftodos,prependcaption,textsize=tiny]{todonotes}
\newcommandx{\unsure}[2][1=]{\todo[linecolor=red,backgroundcolor=red!25,bordercolor=red,#1]{#2}}
\newcommandx{\change}[2][1=]{\todo[linecolor=blue,backgroundcolor=blue!25,bordercolor=blue,#1]{#2}}
\newcommandx{\info}[2][1=]{\todo[linecolor=OliveGreen,backgroundcolor=OliveGreen!25,bordercolor=OliveGreen,#1]{#2}}
\newcommandx{\improvement}[2][1=]{\todo[linecolor=Plum,backgroundcolor=Plum!25,bordercolor=Plum,#1]{#2}}
\newcommandx{\thiswillnotshow}[2][1=]{\todo[disable,#1]{#2}}
%
\newcommand{\norm}[1]{\left\lVert#1\right\rVert}
\DeclareMathOperator*{\argmax}{arg\,max}
\DeclareMathOperator*{\argmin}{arg\,min}

\definecolor{codegreen}{rgb}{0,0.6,0}
\definecolor{codegray}{rgb}{0.35,0.35,0.35}
\definecolor{codepurple}{rgb}{0.58,0,0.82}
\definecolor{backcolour}{rgb}{0.95,0.95,0.92}

\lstdefinestyle{mystyle}{
	backgroundcolor=\color{backcolour},   
	commentstyle=\color{codegreen},
	keywordstyle=\color{magenta},
	numberstyle=\tiny\color{codegray},
	stringstyle=\color{codepurple},
	basicstyle=\footnotesize,
	breakatwhitespace=false,         
	breaklines=true,                 
	captionpos=b,                    
	keepspaces=true,                 
	numbers=left,                    
	numbersep=5pt,                  
	showspaces=false,                
	showstringspaces=false,
	showtabs=false,                  
	tabsize=2
}

\lstset{style=mystyle}

\usepackage{subfigure}
\usepackage{verbatim}
%\usepackage{times}
\usepackage{paralist}
\usepackage{fancyvrb}

\usepackage{enumerate}
\usepackage{enumitem}% http://ctan.org/pkg/enumitem
\usepackage{changepage}


\allowdisplaybreaks

\definecolor{Darkblue}{rgb}{0,0,0.4}
\definecolor{Brown}{cmyk}{0,0.81,1.,0.60}
\definecolor{Purple}{cmyk}{0.45,0.86,0,0}
\newcommand{\mydriver}{hypertex}
\ifpdf
\renewcommand{\mydriver}{pdftex}
\fi
\usepackage[breaklinks,\mydriver]{hyperref}
\hypersetup{colorlinks=true,%pdfborder={1 1 1 [3]},%
	citebordercolor={.6 .6 .6},linkbordercolor={.6 .6 .6},%
	citecolor=Darkblue,urlcolor=blue,linkcolor=Darkblue,pagecolor=blue}
\newcommand{\lref}[2][]{\hyperref[#2]{#1~\ref*{#2}}}

\makeatletter
\setlength{\parindent}{0pt}
\addtolength{\partopsep}{-2mm}
\setlength{\parskip}{5pt plus 1pt}
%\addtolength{\theorempreskipamount}{-1mm}
%\addtolength{\theorempostskipamount}{-1mm}
\addtolength{\abovedisplayskip}{-3mm}
\addtolength{\textheight}{45pt}
\addtolength{\footskip}{-10pt}
\makeatother

\newcommand{\cmark}{\ding{51}}%
\newcommand{\xmark}{\ding{55}}%

\newcommand*{\QEDA}{\hfill\ensuremath{\blacksquare}}%
\newcommand*{\QEDB}{\hfill\ensuremath{\square}}%
\newcommand*{\checkedbox}{$\text{\rlap{$\checkmark$}}\square $\ }%
\DeclareMathOperator*{\argminB}{argmin}   % Jan Hlavacek


\newcommand{\homework}[5]{
	\noindent
	\begin{center}
		\framebox{
			\vbox{
				\hbox to 5.78in { {\bf  } \hfill #2 }
				\vspace{4mm}
				\hbox to 5.78in { {\Large \hfill PROSPERO Application Form  \hfill} }
				\vspace{4mm}
			}
		}
	\end{center}
	\vspace*{4mm}
}

%\newcommand{\lecture}[4]{\handout{#1}{#2}{#3}{Scribe: #4}{Lecture #1}}

% Useful macros
\newcommand{\E}{\mathbb{E}}

\newcommand{\dataitem}[2]{
	\section{#1}
	\textcolor{codegray}{#2} \\
}

\newcommand{\subdataitem}[2]{
	\subsection*{#1}
	\textcolor{codegray}{#2} \\
}
%%%%%%%%%%%%%%%%%%%%%%%%%%%%%%%%%%%%%%%%%%%%%%%%%%%%%%%%%%%%%%%%%%%%%%%%%%%%

\begin{document}
	
	
	\homework{5}{\today}{}{}{}
	Asterisk $*$  indicates mandatory fields.

	\dataitem{*Review title}{Give the title of the review in English}
	
	
	\dataitem{Original language title.}{For reviews in languages other than English, give the title in the original language. This will be displayed with the English language title.}
		
	
	\dataitem{*Anticipated or actual start date.}{Give the date the systematic review started or is expected to start.}
		
		
	\dataitem{*Anticipated completion date.}{Give the date by which the review is expected to be completed.}
		
		
	\dataitem{* Stage of review at time of this submission.}{
		Tick the boxes to show which review tasks have been started and which have been completed. Update this field each time any amendments are made to a published record. Reviews that have started data extraction (at the time of initial submission) are not eligible for inclusion in PROSPERO. If there is later evidence that incorrect status and/or completion date has been supplied, the published PROSPERO record will be marked as retracted.
		This field uses answers to initial screening questions. It cannot be edited until after registration.}
	
	\begin{tabular}{ p{8 cm}p{5.5cm} p{5.5cm}}
		Stage  & Started &Completed \\
		Preliminary searches: & \checkedbox & $\square$ \\
		Piloting of the study selection process: & $\square$  & $\square$ \\
		Formal screening of search results against eligibility criteria: & $\square$  & $\square$ \\
		Data extraction: & $\square$  & $\square$ \\
		Risk of bias (quality) assessment: & $\square$  & $\square$ \\
		Data analysis: & $\square$  & $\square$ \\
	\end{tabular}

	\dataitem{*Named contact.}{The named contact is the guarantor for the accuracy of the information in the register record. This may be any member of the review team.}
	\paragraph{Email salutation (e.g. ``Dr Smith'' or ``Joanne'') for correspondence:}

	\dataitem{*Named contact email.}{Give the electronic email address of the named contact.}

	\dataitem{Named contact address.}{Give the full institutional/organisational postal address for the named contact.}		

	\dataitem{Named contact phone number.}{Give the telephone number for the named contact, including international dialling code.}

	\dataitem{*Organisational affiliation of the review.}{Full title of the organisational affiliations for this review and website address if available. This field may be completed as 'None' if the review is not affiliated to any organisation.}

	\paragraph{Organisation web address:}

	\dataitem{*Review team members and their organisational affiliations.}{Give the personal details and the organisational affiliations of each member of the review team. Affiliation refers to groups or organisations to which review team members belong. NOTE: email and country now MUST be entered for each person, unless you are amending a published record.}

	\dataitem{*Funding sources/sponsors.}{Details of the individuals, organizations, groups, companies or other legal entities who have funded or 		  sponsored the review.}

	\subdataitem{Grant number(s)}{State the funder, grant or award number and the date of award}

	\dataitem{*Conflicts of interest.}{List actual or perceived conflicts of interest (financial or academic).}

	\dataitem{Collaborators}{Give the name and affiliation of any individuals or organisations who are working on the review but who are not listed as review team members. NOTE: email and country must be completed for each person, unless you are amending a published record.}

	\dataitem{*Review question.}{State the review question(s) clearly and precisely. It may be appropriate to break very broad questions down into a series of related more specific questions. Questions may be framed or refined using PI(E)COS or similar where relevant.}

	\dataitem{*Searches.}{State the sources that will be searched (e.g. Medline). Give the search dates, and any restrictions (e.g. language or publication date). Do NOT enter the full search strategy (it may be provided as a link or attachment below.)}		

	\dataitem{URL to search strategy.}{Upload a file with your search strategy, or an example of a search strategy for a specific database, (including the keywords) in pdf or word format. In doing so you are consenting to the file being made publicly accessible. Or provide a URL or link to the strategy. Do NOT provide links to your search results. Alternatively, upload your search strategy to CRD in pdf format. Please note that by doing so you are consenting to the file being made publicly accessibl}

\dataitem{*Condition or domain being studied.}{Give a short description of the disease, condition or healthcare domain being studied in your systematic review.}

\dataitem{* Participants/population.}{Specify the participants or populations being studied in the review. The preferred format includes details of both inclusion and exclusion criteria.}

\dataitem{*Intervention(s), exposure(s).}{Give full and clear descriptions or definitions of the interventions or the exposures to be reviewed. The preferred format includes details of both inclusion and exclusion criteria.}		


\dataitem{*Comparator(s)/control.}{Where relevant, give details of the alternatives against which the intervention/exposure will be compared (e.g. another intervention or a non-exposed control group). The preferred format includes details of both inclusion and exclusion criteria.}

\dataitem{*Types of study to be included.}{Give details of the study designs (e.g. RCT) that are eligible for inclusion in the review. The preferred format includes both inclusion and exclusion criteria. If there are no restrictions on the types of study, this should be stated.}

\dataitem{Context.}{Give summary details of the setting or other relevant characteristics, which help define the inclusion or exclusion criteria.}

\dataitem{ Main outcome(s).}{Give the pre-specified main (most important) outcomes of the review, including details of how the outcome is defined and measured and when these measurement are made, if these are part of the review inclusion criteria.}		

\subdataitem{*Measures of effect}{Please specify the effect measure(s) for you main outcome(s) e.g. relative risks, odds ratios, risk difference, and/or 'number needed to treat.}

\dataitem{*Additional outcome(s).}{List the pre-specified additional outcomes of the review, with a similar level of detail to that required for main outcomes. Where there are no additional outcomes please state ‘None’ or ‘Not applicable’ as appropriate to the review}
\subdataitem{* Measures of effect}{Please specify the effect measure(s) for you main outcome(s) e.g. relative risks, odds ratios, risk difference, and/or 'number needed to treat.}


\dataitem{*Data extraction (selection and coding).}{Describe how studies will be selected for inclusion. State what data will be extracted or obtained. State how this will be done and recorded.}

\dataitem{*Risk of bias (quality) assessment.}{State which characteristics of the studies will be assessed and/or any formal risk of bias/quality assessment tools that will be used.}

\dataitem{ Strategy for data synthesis.}{Describe the methods you plan to use to synthesise data. This must not be generic text but should be specific to your review and describe how the proposed approach will be applied to your data. If metaanalysis is planned, describe the models to be used, methods to explore statistical heterogeneity, and software package to be used.}		


\dataitem{*Analysis of subgroups or subsets.}{State any planned investigation of ‘subgroups’. Be clear and specific about which type of study or participant will be included in each group or covariate investigated. State the planned analytic approach.}

\dataitem{*Type and method of review.}{Select the type of review, review method and health area from the lists below.}
\subsection*{Type of review}
\begin{tabular}{ p{5.5 cm}p{5.5cm} p{5.5cm}}
	\checkedbox Cost effectiveness & \checkedbox Diagnostic &$\square$ Epidemiologic\\
	$\square$ Individual patient data (IPD) meta-analysis &$\square$Intervention&$\square$ Meta-analysis\\
	$\square$ Methodology &$\square$ Narrative synthesis \ &$\square$ Network meta-analysis\\ 
	$\square$ Pre-clinical &$\square$ Prevention \ &$\square$ Prognostic\\
	$\square$ Prospective meta-analysis (PMA) &$\square$ Review of reviews \ &$\square$ Service delivery\\
	$\square$ Synthesis of qualitative studies  &$\square$ Systematic review \ &$\square$ Other\\
\end{tabular}



\subsection*{Health area of the review}
\begin{tabular}{ p{5.5 cm}p{5.5cm} p{5.5cm}}
	\checkedbox Alcohol/substance misuse/abuse & \checkedbox Blood and immune system &$\square$ Cancer\\
	$\square$ Cardiovascular &$\square$ Care of the elderly &$\square$ Child health \\
	$\square$ Complementary therapies &$\square$ COVID-19 &$\square$ Crime and justice \\
	$\square$  Dental &$\square$ Digestive system &$\square$  Ear, nose and throat\\
	$\square$ Education &$\square$ Endocrine and metabolic disorders&$\square$ Eye disorders \\
	$\square$ General interest &$\square$ Genetics &$\square$ Health inequalities/health equity\\
	$\square$ Infections and infestations &$\square$ International development&$\square$ Mental health and behavioural conditions\\
	$\square$  Musculoskeletal&$\square$ Neurological&$\square$ Nursing \\
	$\square$  Obstetrics and gynaecology &$\square$ Oral health&$\square$ Palliative care \\
	$\square$ Perioperative care&$\square$ Physiotherapy&$\square$ Pregnancy and childbirth\\			
	$\square$ Public health (including social determinants of health) &$\square$ Rehabilitation&$\square$ Respiratory disorders\\			
	$\square$ Service delivery&$\square$ Skin disorders&$\square$ Social care\\			
	$\square$ Surgery&$\square$ Tropical Medicine&$\square$ Urological\\			
	$\square$ Wounds, injuries and accidents&$\square$ Violence and abuse& 
\end{tabular}


\dataitem{Language}{Select each language individually to add it to the list below, use the bin icon to remove any added in error.}

\dataitem{*Country.}{Select the country in which the review is being carried out. For multi-national collaborations select all the countries involved.}		


\dataitem{Other registration details.}{Name any other organisation where the systematic review title or protocol is registered (e.g. Campbell, or The Joanna Briggs Institute) together with any unique identification number assigned by them. If extracted data will be stored and made available through a repository such as the Systematic Review Data Repository (SRDR), details and a link should be included here. If none, leave blank.}

\dataitem{Reference and/or URL for published protocol.}{If the protocol for this review is published provide details (authors, title and journal details, preferably in Vancouver format) Add web link to the published protocol. Or, upload your published protocol here in pdf format. Note that the upload will be publicly accessible. Please note that the information required in the PROSPERO registration form must be completed in full even if access to a protocol is given.}

\dataitem{Dissemination plans.}{Do you intend to publish the review on completion?}
\textcolor{codegray}{Give brief details of plans for communicating review findings?} \\


\dataitem{Keywords}{Give words or phrases that best describe the review. Separate keywords with a semicolon or new line. Keywords help PROSPERO users find your review (keywords do not appear in the public record but are included in searches). Be as specific and precise as possible. Avoid acronyms and abbreviations unless these are in wide use.}		


\dataitem{Details of any existing review of the same topic by the same authors.}{If you are registering an update of an existing review give details of the earlier versions and include a full bibliographic reference, if available.}

\dataitem{*Current review status.}{Update review status when the review is completed and when it is published.New registrations must be ongoing so this field is not editable for initial submission. Please provide anticipated publication date}

\dataitem{Any additional information.}{Provide any other information relevant to the registration of this review.}

\dataitem{Details of final report/publication(s) or preprints if available.}{Leave empty until publication details are available OR you have a link to a preprint (NOTE: this field is not editable for initial submission). List authors, title and journal details preferably in Vancouver format. Give the link to the published review or preprint.}		
		

\end{document}

